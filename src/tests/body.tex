\chapter{Introdução}

Escrever um trabalho científico pode ser uma tarefa desafiadora. \cite{severino}
destaca a complexidade e o rigor necessários na elaboração de trabalhos científicos, que não
apenas envolvem o domínio do conteúdo específico, mas também a aderência às normas
técnicas para apresentação formal e formatação correta.

A Associação Brasileira de Normas Técnicas 
(\acrshort{abnt})
, é a entidade responsável por,
dentre outras, fornecer as normas que regulam o processo de criação de trabalhos acadêmicos.
A Norma Brasileira Regulamentadora 
(\acrshort{nbr})
 Nº 14724, por exemplo: Especifica os princípios
gerais para a elaboração de (teses, dissertações e outros), visando sua apresentação à
instituição (banca, comissão examinadora de professores, especialistas designados e/ou
outros)
\cite{abnt}.

Ademais, ainda com respeito aos trabalhos acadêmicos, não somente a
regulamentação da 
\acrshort{nbr}
14724 deve ser observada. Há ainda a 
\acrshort{nbr}
6023 que trata a respeito
da elaboração de referências e a 
\acrshort{nbr}
10520, que diz respeito às citações em documentos.

\cite{castro}, adverte que: "Em ciência, não pode haver uma
separação entre forma e conteúdo. Trata-se de uma separação fictícia, pois fica se conhecendo
o conteúdo pela forma." Ou seja: A forma do trabalho, sua apresentação, sua formatação e
todo o seu arranjo gráfico é tão importante quanto seu conteúdo. 
\cite{medeiros} vai
complementar essa visão, afirmando que a presentação gráfica "[...] contribui para a
consecução de um trabalho capaz de atingir seu objetivo. Monografia realizada sem a
preocupação gráfica, em geral, acaba malsucedida."

Em seu artigo, 
\cite{SilvaVitoria}
vão analisar as percepções e dificuldades dos
alunos de um curso superior em Tecnologia de Gestão em Recursos Humanos. Dentre suas
dificuldades, (dos alunos em questão), é destacada a questão da formatação do trabalho
acadêmico. Há também o fato de que as bancas avaliam os trabalhos baseadas em critérios da
própria Instituição de Ensino Superior (
\acrshort{ies}
), critérios estes que não estão necessariamente
presentes nas normas da \acrshort{abnt}, ou seja, há uma subjetividade presente que não é comum a
todas às \acrshort{ies} quanto a questão da formatação. Essa subjetividade contribui para a confusão dos
alunos, pois a \acrshort{ies} avaliará de acordo com aquilo que julga apropriado, o que muitas vezes
pode obscurecer o direcionamento do aluno ao redigir/formatar seu trabalho."

\clearpage

\cite{santos}
em seu Trabalho de Conclusão de Curso
(\acrshort{tcc})
, também analisa as
dificuldades encontradas por egressos, desta vez do curso de Ciências Contábeis da
Universidade Federal da Paraíba
(\acrshort{ufpb}).
Em sua pesquisa é destacado que "Quanto a
formatação do trabalho com as normas da 
\acrshort{abnt}, [...], 60\% teve alguma dificuldade, inclusive
32\% teve muita dificuldade." Ou seja, a formatação do trabalho é um grande desafio presente
na vida de boa parte dos estudantes em processo de escrita.

\section{Objetivo}

Levando em consideração os problemas que os alunos de diversas instituições de ensino enfrentam ao elaborar seus respectivos
trabalhos (conforme apresentado acima), o objetivo deste instrumento é desenvolver uma plataforma web de alta
interatividade\footnote{Refere-se à capacidade de um sistema, aplicação ou interface de responder
        às ações do usuário de maneira eficaz e intuitiva}
e
inteligibilidade\footnote{Refere-se à clareza e compreensibilidade da interface, documentação e feedback fornecidos pelo
    sistema. Um software inteligível facilita o entendimento do usuário sobre como utilizá-lo e quais são os resultados de suas ações.},
de modo que o discente possa se preocupar apenas com o conteúdo. Os detalhes de formatação, de acordo com os padrões da
\acrshort{abnt}
e da
\acrshort{ies},
ficarão a cargo da própria plataforma.



\section{Pilares da aplicação}

\section{Resultados}