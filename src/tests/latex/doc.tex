\documentclass[12pt,a4paper,oneside,brazil]{abntex2}
\usepackage[utf8]{inputenc}
\usepackage[T1]{fontenc}
\usepackage[brazil]{babel}
\usepackage{graphicx}
\usepackage{lipsum} % Pacote para gerar texto fictício
\usepackage{helvet}
\usepackage{ragged2e}
\usepackage{glossaries}
\usepackage[alf]{abntex2cite} % Citações padrão ABNT
\usepackage{indentfirst}
\usepackage{xpatch}
\usepackage{tabularx} % For flexible columns
\usepackage{blindtext}  % This package provides \blindtext
\usepackage{titlesec}   % This package is used for custom chapter title formatting
\usepackage{float}
\usepackage{fancyvrb}
\usepackage{listings}
\usepackage{caption}
\usepackage{acronym}

\setlength{\parindent}{1.5cm}

\renewcommand{\familydefault}{\sfdefault}

\hypersetup{
    hidelinks
}

\titulo{Plataforma de Edição/Automação para trabalhos acadêmicos}
\autor{Higor Ferreira Alves Santos}
\orientador{Marcelo Antônio Adad}

\newcommand{\ies}{Pontifícia Universidade Católica de Goiás}
\newcommand{\escola}{Escola Politécnica e de Artes}
\newcommand{\curso}{Engenharia de Computação}

\newcommand{\grauOrientador}{Prof. M.E.E.}
\newcommand{\grauAluno}{Bacharel}

\newcommand{\grauBancaUm}{Prof. Dra.}
\newcommand{\bancaUm}{Miriam Gusmão}
\newcommand{\grauBancaDois}{}
\newcommand{\bancaDois}{}

\instituicao{%
    \ies
    \par
    \escola
    \par
    \curso
}
\local{GOIÂNIA - GO}
\data{2023}

\newglossary*{abreviacao}{Lista de abreviaturas}
\newglossary*{sigla}{Lista de siglas}
\newglossary*{simbolo}{Lista de símbolos}

\newacronym[type=sigla]{abnt}{ABNT}{Associação Brasileira de Normas Técnicas}
\newacronym[type=sigla]{css}{CSS}{Cascading Style Sheet}
\newacronym[type=sigla]{cms}{CMS}{Content Management System}
\newacronym[type=sigla]{ecma}{ECMA}{European Computer Manufacturers Association}
\newacronym[type=sigla]{xhtml}{XHTML}{eXtensible HyperText Markup Language}
\newacronym[type=sigla]{xml}{XML}{eXtensible Markup Language}
\newacronym[type=sigla]{ies}{IES}{Instituição de Ensino Superior}
\newacronym[type=sigla]{mee}{M.E.E}{Mestre em Engenharia Elétrica}
\newacronym[type=sigla]{nbr}{NBR}{Norma Brasileira Regulamentadora}
\newacronym[type=sigla]{pucgo}{PUC-GO}{Pontifícia Universidade Católica de Goiás}
\newacronym[type=sigla]{pdf}{PDF}{Portable Document Format}
\newacronym[type=sigla]{svg}{SVG}{Scalable Vector Graphics}
\newacronym[type=sigla]{tcc}{TCC}{Trabalho de Conclusão de Curso}
\newacronym[type=sigla]{ufpb}{UFPB}{Universidade Federal da Paraíba}
\newacronym[type=abreviacao]{cern}{CERN}{Conseil Européen pour la Recherche Nucléaire}
\newacronym[type=abreviacao]{xss}{XSS}{Cross-Site Scripting}
\newacronym[type=abreviacao]{html}{HTML}{HyperText Markup Language}
\newacronym[type=abreviacao]{php}{PHP}{Hypertext Preprocessor}
\newacronym[type=abreviacao]{http}{HTTP}{Hypertext Transfer Protocol}
\newacronym[type=abreviacao]{js}{JS}{JavaScript}
\newacronym[type=abreviacao]{json}{JSON}{JavaScript Object Notation}
\newacronym[type=abreviacao]{mathml}{MathML}{Mathematical Markup Language}
\newacronym[type=abreviacao]{prof}{Prof}{Professor}
\newacronym[type=abreviacao]{web}{Web}{World Wide Web}
\newacronym[type=abreviacao]{w3c}{W3C}{World Wide Web Consortium}


\makeglossaries
\DefineVerbatimEnvironment{GitVersionCode}{Verbatim}
{numbers=left, numbersep=8pt, frame=single, framerule=0.5pt, firstnumber=1}
\DefineVerbatimEnvironment{createNextJsCommand}{Verbatim}
{numbers=left, numbersep=8pt, frame=single, framerule=0.5pt, firstnumber=1}
\DefineVerbatimEnvironment{promptNextJs}{Verbatim}
{numbers=left, numbersep=8pt, frame=single, framerule=0.5pt, firstnumber=1}
\DefineVerbatimEnvironment{yarnAddDpts}{Verbatim}
{numbers=left, numbersep=8pt, frame=single, framerule=0.5pt, firstnumber=1}
\DefineVerbatimEnvironment{layoutExample}{Verbatim}
{numbers=left, numbersep=8pt, frame=single, framerule=0.5pt, firstnumber=1}
\DefineVerbatimEnvironment{pageExample}{Verbatim}
{numbers=left, numbersep=8pt, frame=single, framerule=0.5pt, firstnumber=1}
\DefineVerbatimEnvironment{5f9422f8f10b49ca92a90ac505d8dbd2}{Verbatim}
{numbers=left, numbersep=8pt, frame=single, framerule=0.5pt, firstnumber=1}
\DefineVerbatimEnvironment{Code3bf6e8092d5d43cf908d6217154e1cbe}{Verbatim}
{numbers=left, numbersep=8pt, frame=single, framerule=0.5pt, firstnumber=1}
\DefineVerbatimEnvironment{Codea6e986c21cb04e17a0b584a4b83f0255}{Verbatim}
{numbers=left, numbersep=8pt, frame=single, framerule=0.5pt, firstnumber=1}
\DefineVerbatimEnvironment{Code65951ff4191f43b5a147cbd2d29d0997}{Verbatim}
{numbers=left, numbersep=8pt, frame=single, framerule=0.5pt, firstnumber=82}
\DefineVerbatimEnvironment{Coded77e6ad668474c82be18e2f3d51a7705}{Verbatim}
{numbers=left, numbersep=8pt, frame=single, framerule=0.5pt, firstnumber=120}
\DefineVerbatimEnvironment{Code576b60c745934ed8991a228e4f7c7891}{Verbatim}
{numbers=left, numbersep=8pt, frame=single, framerule=0.5pt, firstnumber=75}
\DefineVerbatimEnvironment{Code6815116651b24444a11d0596ad342c51}{Verbatim}
{numbers=left, numbersep=8pt, frame=single, framerule=0.5pt, firstnumber=116}
\DefineVerbatimEnvironment{Code55686b25a2b44b728eb3c0190bd78e4e}{Verbatim}
{numbers=left, numbersep=8pt, frame=single, framerule=0.5pt, firstnumber=136}
\DefineVerbatimEnvironment{Code4f8a1cb5accd4fccb936cc5db4dc36fc}{Verbatim}
{numbers=left, numbersep=8pt, frame=single, framerule=0.5pt, firstnumber=208}
\DefineVerbatimEnvironment{Codeb1f05e50235e4907b05b96b12c2ea861}{Verbatim}
{numbers=left, numbersep=8pt, frame=single, framerule=0.5pt, firstnumber=172}
\DefineVerbatimEnvironment{Codedb32e4b19e8445d5a93830fc1a3e4566}{Verbatim}
{numbers=left, numbersep=8pt, frame=single, framerule=0.5pt, firstnumber=50}
\DefineVerbatimEnvironment{Code3ebe71a2cd68490eba49433770b1fd72}{Verbatim}
{numbers=left, numbersep=8pt, frame=single, framerule=0.5pt, firstnumber=16}
\DefineVerbatimEnvironment{Code5359def56d0b416692f1d2310024ecd2}{Verbatim}
{numbers=left, numbersep=8pt, frame=single, framerule=0.5pt, firstnumber=35}
\DefineVerbatimEnvironment{Code9de9e761a7f64a079b8b7e219f13ac8f}{Verbatim}
{numbers=left, numbersep=8pt, frame=single, framerule=0.5pt, firstnumber=51}
\DefineVerbatimEnvironment{codeEscape}{Verbatim}
{numbers=left, numbersep=8pt, frame=single, framerule=0.5pt, firstnumber=1}
\DefineVerbatimEnvironment{processHTML}{Verbatim}
{numbers=left, numbersep=8pt, frame=single, framerule=0.5pt, firstnumber=1}
\DefineVerbatimEnvironment{processHTML2}{Verbatim}
{numbers=left, numbersep=8pt, frame=single, framerule=0.5pt, firstnumber=39}
\DefineVerbatimEnvironment{posProcess1}{Verbatim}
{numbers=left, numbersep=8pt, frame=single, framerule=0.5pt, firstnumber=1}
\DefineVerbatimEnvironment{ParagraphBlockCode}{Verbatim}
{numbers=left, numbersep=8pt, frame=single, framerule=0.5pt, firstnumber=37}
\DefineVerbatimEnvironment{HeaderBlockCode}{Verbatim}
{numbers=left, numbersep=8pt, frame=single, framerule=0.5pt, firstnumber=5}
\DefineVerbatimEnvironment{getParagraphCode}{Verbatim}
{numbers=left, numbersep=8pt, frame=single, framerule=0.5pt, firstnumber=1}
\DefineVerbatimEnvironment{getHeaderCode}{Verbatim}
{numbers=left, numbersep=8pt, frame=single, framerule=0.5pt, firstnumber=1}
\DefineVerbatimEnvironment{getImageCode1}{Verbatim}
{numbers=left, numbersep=8pt, frame=single, framerule=0.5pt, firstnumber=1}
\DefineVerbatimEnvironment{getImageCode2}{Verbatim}
{numbers=left, numbersep=8pt, frame=single, framerule=0.5pt, firstnumber=13}
\DefineVerbatimEnvironment{getListCode1}{Verbatim}
{numbers=left, numbersep=8pt, frame=single, framerule=0.5pt, firstnumber=1}
\DefineVerbatimEnvironment{getPageBreak1}{Verbatim}
{numbers=left, numbersep=8pt, frame=single, framerule=0.5pt, firstnumber=1}
\DefineVerbatimEnvironment{Code20ee6ec7cc804ebc968d38486325deca}{Verbatim}
{numbers=left, numbersep=8pt, frame=single, framerule=0.5pt, firstnumber=1}
\DefineVerbatimEnvironment{Code19ad897f44334945a96d75ebe98275bb}{Verbatim}
{numbers=left, numbersep=8pt, frame=single, framerule=0.5pt, firstnumber=1}
\DefineVerbatimEnvironment{Code0264bd9370f04cc2a092f539cb2d7205}{Verbatim}
{numbers=left, numbersep=8pt, frame=single, framerule=0.5pt, firstnumber=3}
\DefineVerbatimEnvironment{Codec07f9c01965a46e093d4b5a3e8f9d306}{Verbatim}
{numbers=left, numbersep=8pt, frame=single, framerule=0.5pt, firstnumber=64}
\DefineVerbatimEnvironment{Code6fa234f7ce9a48158b541af61dea511b}{Verbatim}
{numbers=left, numbersep=8pt, frame=single, framerule=0.5pt, firstnumber=1}

% Definindo um comando personalizado para o tamanho da fonte
\newcommand{\chaptersize}{\fontsize{12pt}{1.5}\selectfont}
% Definindo a formatação do capítulo
\titleformat{\chapter}{\normalfont\bfseries\chaptersize\MakeUppercase}{\thechapter}{1em}{}

\titleformat{\section}{\normalfont\bfseries\chaptersize}{\thesection}{1em}{}

\titleformat{\subsection}{\normalfont\itshape\bfseries\chaptersize}{\thesubsection}{1em}{}

\titleformat{\subsection}{\normalfont\itshape\bfseries\chaptersize}{\thesubsection}{1em}{}

\titleformat{\subsubsection}{\normalfont\chaptersize}{\thesubsubsection}{1em}{}

\titleformat{\subsubsubsection}{\normalfont\chaptersize}{\thesubsubsubsection}{1em}{}

% Alterando o título da lista de figuras
\renewcommand{\listfigurename}{Lista de Figuras}
% Alterando o título da lista de tabelas
\renewcommand{\listtablename}{Lista de Tabelas}
% Alterando o título do sumário
\renewcommand{\contentsname}{Sumário}

% Início do documento
\begin{document}

% Capa personalizada
\begin{capa}
    \center

    \OnehalfSpacing
    \ABNTEXchapterfont\bfseries{\textsc{\MakeUppercase{\imprimirinstituicao}}}

    \vfill

    % Incluir logotipo
    \includegraphics[width=0.15\textwidth]{/home/higor/Documents/TCC/editor2/src/assets/logo.png}

    \vfill

    \ABNTEXchapterfont\bfseries{\MakeUppercase{\imprimirtitulo}}

    \vfill

    \MakeUppercase{\imprimirautor}

    \vfill

    \bfseries{\MakeUppercase{\imprimirlocal}}

    \bfseries{\MakeUppercase{\imprimirdata}}
\end{capa}

% Folha de rosto personalizada
\begin{folhaderosto}

    \centering

    \MakeUppercase{\imprimirautor}

    \vfill

    \ABNTEXchapterfont\bfseries{\MakeUppercase{\imprimirtitulo}}

    \vfill

    \justifying
    \noindent\hspace*{70mm}%
    \begin{minipage}{\dimexpr\textwidth-70mm}
        \textnormal{
            Trabalho de Conclusão de Curso apresentado à
            \escola, da \ies, como parte dos
            requisitos para a obtenção do título de \grauAluno\,em
            \curso.\\\\
            Orientador:\\
            \begin{flushright}
                \grauOrientador \imprimirorientador
            \end{flushright}
            Banca examinadora:\\
            \begin{flushright}
                \grauBancaUm \, \bancaUm
                \grauBancaDois \, \bancaDois
            \end{flushright}
        }
    \end{minipage}

    \vfill

    \centering

    \bfseries{\imprimirlocal}

    \bfseries{\imprimirdata}
\end{folhaderosto}

% Folha de aprovação
\clearpage
    \centering
    \MakeUppercase{\imprimirautor}

    \vfill

    \ABNTEXchapterfont\bfseries{\MakeUppercase{\imprimirtitulo}}

    \vfill

    \justifying

    \textnormal{Trabalho de Conclusão de Curso aprovado em sua forma final pela Escola Politécnica e de
    Artes, da Pontifícia Universidade Católica de Goiás, para obtenção do título de \grauAluno\,em
    Engenharia de Computação, em: \rule{8mm}{0.4pt}/ \rule{8mm}{0.4pt}/ \rule{16mm}{0.4pt}}

    \centering

    \vspace*{3cm}

    \begin{flushright}
    \rule{10cm}{0.4pt}\\
    \textnormal{Orientador: \grauOrientador \imprimirorientador}

    \vspace*{10mm}

    \rule{10cm}{0.4pt}\\
    \textnormal{Orientador1: \grauBancaUm \, \bancaUm}

    \vspace*{10mm}

    \rule{10cm}{0.4pt}\\
    \textnormal{Orientador2: \grauBancaDois \, \bancaDois}
    \end{flushright}

    \vspace*{6cm}

    \bfseries{\imprimirlocal}

    \bfseries{\imprimirdata}
\clearpage % Começa uma nova página para a folha de rosto

%Dedicatória
\centering
\ABNTEXchapterfont\bfseries{\textsc{\MakeUppercase{Dedicatória}}}\\
\vspace*{3cm}
\justifying
\normalfont
Dedico a Aquele que se assenta inerte no trono de glória, em algum lugar
remoto do Universo.

À Pátria amada,
vestida em sua pele de cordeiro. Que promove a bondade
e a justiça do alto de suas majestosas pirâmides sociais, sustentadas
sob os pilares de sal da ambivalência.


À minha família, autora das maiores alegrias e melancolias que a
vida pode trazer.

\vspace{10mm}

A Deus, à Pátria, à Família!

\vspace{20mm}

Ao meu pai: Euripedes Alves dos Santos; minha mãe: Maria
Aparecida Ferreira Gomes dos Santos; e minha irmã:
Sthefany Ferreira Alves dos Santos.

\vspace{10mm}

Em memória do meu avô: Gabriel Ferreira Gomes, falecido durante
o processo de escrita deste trabalho.

\vspace{10mm}

Em memória de minhas avós: Maria Tavares dos Santos e
Jesuína Pereira Gomes.

\vspace{10mm}

Às amizades de qualidade conquistadas, cujo julgo compartilhado
torna a caminhada mais leve.

\vspace{30mm}

(Melhorar isso aqui)
Às criaturas do Vale, cuja matéria de acolhimento e aceitação
as vezes é tão boa quanto as extremas opostas criaturas do Caminho.



\clearpage

%Agradecimentos
\centering
\ABNTEXchapterfont\bfseries{\textsc{\MakeUppercase{Agradecimetos}}}\\
\vspace*{3cm}
\justifying
\normalfont
% \lipsum[1]
\clearpage

%Epígrafe
\centering
\ABNTEXchapterfont\bfseries{\textsc{\MakeUppercase{Epígrafe}}}\\
\vspace*{3cm}
\justifying
\normalfont
% \lipsum[1-3]
\clearpage

%Resumo
\centering
\ABNTEXchapterfont\bfseries{\textsc{\MakeUppercase{Resumo}}}\\
\vspace*{3cm}
\justifying
\normalfont
% \lipsum[12-13]
\clearpage

%Abstract
\centering
\ABNTEXchapterfont\bfseries{\textsc{\MakeUppercase{Abstract}}}\\
\vspace*{3cm}
\justifying
\normalfont
% \lipsum[7]
\clearpage

\pagenumbering{roman}

\listoffigures   % Lista de Figuras
\clearpage

\listoftables    % Lista de Tabelas
\clearpage


\printglossary[type=abreviacao,title=Lista de Abreviaturas]                    
\clearpage

\printglossary[type=sigla,title=Lista de Siglas]                    
\clearpage

% Correção: Usar \tableofcontents para o Sumário
\tableofcontents
\clearpage



\pagenumbering{arabic}
\setcounter{page}{1}
\textual

\justifying
\normalfont


\chapter{Intrudução}

Escrever um trabalho científico pode ser uma tarefa desafiadora. \cite{severino}
destaca a complexidade e o rigor necessários na elaboração de trabalhos científicos, que não
apenas envolvem o domínio do conteúdo específico, mas também a aderência às normas
técnicas para apresentação formal e formatação correta.

A Associação Brasileira de Normas Técnicas 
(\acrshort{abnt})
, é a entidade responsável por,
dentre outras, fornecer as normas que regulam o processo de criação de trabalhos acadêmicos.
A Norma Brasileira Regulamentadora 
(\acrshort{nbr})
 Nº 14724, por exemplo: Especifica os princípios
gerais para a elaboração de (teses, dissertações e outros), visando sua apresentação à
instituição (banca, comissão examinadora de professores, especialistas designados e/ou
outros) (ABNT, 2011).


\bibliography{referencias}

\end{document}