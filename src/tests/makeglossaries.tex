\newglossary*{abreviacao}{Lista de abreviaturas}
\newglossary*{sigla}{Lista de siglas}
\newglossary*{simbolo}{Lista de símbolos}
 \newacronym[type=sigla]{abnt}{ABNT}{Associação Brasileira de Normas Técnicas}
\newacronym[type=sigla]{css}{CSS}{Cascading Style Sheet}
\newacronym[type=sigla]{cern}{CERN}{Conseil Européen pour la Recherche Nucléaire}
\newacronym[type=sigla]{cms}{CMS}{Content Management System}
\newacronym[type=sigla]{ecma}{ECMA}{European Computer Manufacturers Association}
\newacronym[type=sigla]{xhtml}{XHTML}{eXtensible HyperText Markup Language}
\newacronym[type=sigla]{xml}{XML}{eXtensible Markup Language}
\newacronym[type=sigla]{html}{HTML}{HyperText Markup Language}
\newacronym[type=sigla]{http}{HTTP}{Hypertext Transfer Protocol}
\newacronym[type=sigla]{ies}{IES}{Instituição de Ensino Superior}
\newacronym[type=sigla]{mee}{M.E.E}{Mestre em Engenharia Elétrica}
\newacronym[type=sigla]{nbr}{NBR}{Norma Brasileira Regulamentadora}
\newacronym[type=sigla]{pucgo}{PUC-GO}{Pontifícia Universidade Católica de Goiás}
\newacronym[type=sigla]{pdf}{PDF}{Portable Document Format, (Formato  de Documento Portável)}
\newacronym[type=sigla]{svg}{SVG}{Scalable Vector Graphics}
\newacronym[type=sigla]{tcc}{TCC}{Trabalho de Conclusão de Curso}
\newacronym[type=sigla]{ufpb}{UFPB}{Universidade Federal da Paraíba}
\newacronym[type=abreviacao]{XSS}{XSS}{Cross-Site Scripting}
\newacronym[type=abreviacao]{PHP}{PHP}{Hypertext Preprocessor}
\newacronym[type=abreviacao]{JS}{JS}{JavaScript}
\newacronym[type=abreviacao]{json}{JSON}{JavaScript Object Notation, (Notação de Objeto JavaScript)}
\newacronym[type=abreviacao]{MathML}{MathML}{Mathematical Markup Language}
\newacronym[type=abreviacao]{Prof}{Prof}{Professor}
\newacronym[type=abreviacao]{W3C}{W3C}{World Wide Web Consortium}

 % Define a custom glossary style without page numbers
\newglossarystyle{grid}{%
    \setglossarystyle{list}% base this style on the list style
    \renewcommand*{\glossentry}[2]{%
        \begin{tabularx}{\textwidth}{@{}p{0.2\textwidth} p{0.8\textwidth}@{}}
            \textbf{\glossentryname{##1}} & \glossentrydesc{##1}%
        \end{tabularx}%
    }%
}
 % Apply the custom glossary style to each glossary
\setglossarystyle{grid}
 \makeglossaries
 % Redefinição do comando \glossarysection para personalizar o título
\renewcommand{\glossarysection}[2][]{%
    \begin{center} % centraliza o título
    \section*{\normalfont\fontsize{12}{14}\bfseries\selectfont #2} % título com fonte de 12pt, em negrito
    \end{center}
    \addcontentsline{toc}{section}{#2} % adiciona ao sumário
    \markboth{#2}{#2} % marcação para cabeçalho
    \vspace{15mm} % espaçamento após o título
}